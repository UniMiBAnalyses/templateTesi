\documentclass[12pt,twoside,a4paper,fleqn,openright]{book}

% 11pt         e' la dimensione del carattere
% twoside      e' per avere fronte/retro
% a4paper      si spiega da solo
% fleqn        e' per avere le equazioni allineate a sx
% openright    e' per iniziare i capitoli sempre a destra

% \usepackage[english,italian]{babel}  % per la lingua: cambia il nome dei capitoli in italiano
\usepackage[dvips]{graphicx}         % per inserire le immagini
\usepackage{verbatim}                % per usare \begin{comment}
\usepackage{syntonly}                % per usare il comando syntaxonly
\usepackage{subfigure}               % per strutturare le immagni
\usepackage{fancyheadings}           % per la righetta fottuta & co.
%\usepackage{rotating}                % chissa' a che serve?


%syntaxonly       % per eseguire solo un controllo sintattico, senza compilare
\linespread{1.2}  % Per avere una maggiore interlinea. Se si dovesse cambiare, attenzione ai 
                  % comandi dati per l'interlinea delle caption. 

\author{Mario Rossi}
\title{TESI DI LAUREA}



\begin{document}



%\pagestyle{headings}

\maketitle

\pagestyle{empty}

\tableofcontents


\pagestyle{fancyplain}
\addtolength{\headwidth}{\marginparsep}
\addtolength{\headwidth}{\marginparwidth}
% remember chapter title
\renewcommand{\chaptermark}[1]{\markboth{{\chaptername\ \thechapter}\ \--- #1}{}} %{\markboth{#1}{#1}}
% section number and title
\renewcommand{\sectionmark}[1]{\markright{\thesection\ \ #1}}
\lhead[\fancyplain{}{\thepage}]{\fancyplain{}{\emph\rightmark}}
\rhead[\fancyplain{}{\emph\leftmark}]{\fancyplain{}{\thepage}}
\cfoot{}

\chapter*{Introduction}
\addcontentsline{toc}{chapter}{Introduction}{}  % mette l'introduzione nell'indice

The first part of the introduction is the abstract of the thesis, 
which contains an itroductive summary of the whole work, 
without the actual results.
The summary must contain only the main points, 
leaving to the thesis description all the details of the implementation,
of the techniques used, 
of specific effects.
In the introduction, only the main physics motivation
and the key elements of the study need to be reported.

The second part of the introduction
is the list of chapters presented later on, 
with one-paragraph description of the content of each of them, 
such that the reader is aware of what to expect.

A useful introduction on how to use \LaTeXe{} can be found at this reference~\cite{NSS}.
  
\chapter{Title of the First Chapter}


\section{The main structure of a thesis}

A thesis may be ideally thought of as divided into three parts.
The first chapter contains the description of the environment where the work takes place,
both theoretical and experimental.
Both of them are instrumental to the work presented in the thesis, 
and are not meant to replace any notions that the reader would rather acquire 
from the existing literature.
The target audience of this chapter are physicists,
who are well aware of the main principle of the modern physics,
threfore the is meant only to address the specific knowledge to understand the work presented
in the following chapters,
in terms of experimental situation and theoretical motivations.
This may for example include a brief description of the Large Hadron Collider
and of the CMS detector, for the benefit of colleagues working on other subjects
for what concerns the experimental part; 
of the generic particle content of the Standard Model,
its limitations if needed,
and the connection to the aims of the work.


\section{Consistency}

The thesis will be written with a consistent style,
that should be maintanined across the whole document.
\begin{itemize}
\item the writer may choose whether to used the first-person singular, 
      the fist-person plural, or an impersonal form in the text, 
      and then will stick to the choice
\item relevant concepts, objects, tools, algorithms shall always have the same name,
      in order not to confuse the reader, even if this generates syntactic repetitions
\item when used, acronyms shall be at the first occurence in the text, for example
      ``Large Hadron Collider (LHC)'' and then always used in the acronym form
\end{itemize}	


\section{Completeness of Information}

The information in the thesis shall be complete,
giving the reader all the technical details necessary for the understanding of each reported result.
References to the thesis itself, like "as written before" or such, shall be avoided,
since it's understood that the reader already read what preceeds each sentence,
or is able to browse the table of contents to identify the needed piece of information.
\newline{}
At the same time,
this does not imply that the thesis is a narration of the thesis work:
the focus is on the results, not on the history of their achievement.
For example,
failed attempts shall not be reported if they do not constitute a relevant scientific piece of information.


\section{Images and Tables}

The writer may use tables and figures in the writing,
remembering that each of them should be always mentioned explicitly in the main text with its numbering,
and that each of them should have a caption long and clear enough 
to allow the reader to understand what is presented 
without the need of searching for the reference in the main text.
    
\chapter{Title of the Second Chapter}

The second part in which a thesis is divided into
contains the specific introduction to the work performed.
The specific physics context shall be described,
for example in terms of what process has been studied,
what are its characteristics,
peculiarities
and main challenges in its study.
From the experimental point of view, 
a description will be presented for 
the data taking conditions, 
the particle reconstruction,
or the events simulation.
Concerning the extraction of the results,
the tools utilised will be presented, 
for example when used to isolate signal over background,
or to fit distributions to extract the final result,
or to develop the best reconstruction for a given quantity of interest.

    
\chapter{Title of the Third Chapter}

The third part of the thesis is the place where the actual work gets described.
Here, 
the reader shall be guided to the final result obtained,
which shall have the main relevance of the text.
By no means the content exposed here should be weighed 
by the effort spent in producing each intermediate result,
not in terms of the number of plots, 
nor in terms of several failed attempts 
(but if the failure is relevant for the final result).
The description of the work shall be complete,
to guide the reader in the understanding of all the elements
necessary to grasp the final result obtained.

As a matter of fact, 
in the thesis writing one should best start by compiling the index of the work,
in terms of chapters and subchapters,
with a single sentence, for each subchapter, 
describing what is expected to be containted in that section.
To decide what is relevant to be put in the text,
one possible approach is to start from the final result description
and proceed conceptually backward,
adding in each section what is needed there to understand the result, 
or the content present in the part following it.
The aim of this procedure is to avoid adding unnecessary information to the thesis,
while not forgetting relevant bits of it.

One final section of the third chapter may contain 
the implications of the results,
the next steps to be undertaken,
and future prospects of the study,
from the point of view of the writer.    
\chapter*{Conclusions}
\addcontentsline{toc}{chapter}{Conclusions}{}  % mette le conclusioni nell'indice

The final chapter of the thesis
is a summary of the work done.
Therefore, in its first part it resembles much the introduction,
adding to it the actual result of the work,
its future evolution and prospects,
in the view of the writer.         
% The bibliography contains all the bibliographic references presented in the text.

\begin{thebibliography}{99} % il 99 e' il massimo numero che ci si aspetta di leggere fra i riferimenti

\bibitem{NSS}
T.~Oetiker, H.~Partl, I.~Hyna, and E.~Schlegl,
``The Not So Short Introduction to \LaTeXe'',
https://tobi.oetiker.ch/lshort/lshort.pdf

\end{thebibliography}







         

% INIZIO COMMENTO
\begin{comment} 

% quello che sta qui non viene compilato

\listoffigures

\listoftables

\end{comment}
% FINE COMMENTO


\end{document}




















