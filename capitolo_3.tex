\chapter{Title of the Third Chapter}

The third part of the thesis is the place where the actual work gets described.
Here, 
the reader shall be guided to the final result obtained,
which shall have the main relevance of the text.
By no means the content exposed here should be weighed 
by the effort spent in producing each intermediate result,
not in terms of the number of plots, 
nor in terms of several failed attempts 
(but if the failure is relevant for the final result).
The description of the work shall be complete,
to guide the reader in the understanding of all the elements
necessary to grasp the final result obtained.

As a matter of fact, 
in the thesis writing one should best start by compiling the index of the work,
in terms of chapters and subchapters,
with a single sentence, for each subchapter, 
describing what is expected to be containted in that section.
To decide what is relevant to be put in the text,
one possible approach is to start from the final result description
and proceed conceptually backward,
adding in each section what is needed there to understand the result, 
or the content present in the part following it.
The aim of this procedure is to avoid adding unnecessary information to the thesis,
while not forgetting relevant bits of it.

One final section of the third chapter may contain 
the implications of the results,
the next steps to be undertaken,
and future prospects of the study,
from the point of view of the writer.