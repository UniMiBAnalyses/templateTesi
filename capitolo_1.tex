\chapter{Title of the First Chapter}


\section{The main structure of a thesis}

A thesis may be ideally thought of as divided into three parts.
The first chapter contains the description of the environment where the work takes place,
both theoretical and experimental.
Both of them are instrumental to the work presented in the thesis, 
and are not meant to replace any notions that the reader would rather acquire 
from the existing literature.
The target audience of this chapter are physicists,
who are well aware of the main principle of the modern physics,
threfore the is meant only to address the specific knowledge to understand the work presented
in the following chapters,
in terms of experimental situation and theoretical motivations.
This may for example include a brief description of the Large Hadron Collider
and of the CMS detector, for the benefit of colleagues working on other subjects
for what concerns the experimental part; 
of the generic particle content of the Standard Model,
its limitations if needed,
and the connection to the aims of the work.


\section{Consistency}

The thesis will be written with a consistent style,
that should be maintanined across the whole document.
\begin{itemize}
\item the writer may choose whether to used the first-person singular, 
      the fist-person plural, or an impersonal form in the text, 
      and then will stick to the choice
\item relevant concepts, objects, tools, algorithms shall always have the same name,
      in order not to confuse the reader, even if this generates syntactic repetitions
\item when used, acronyms shall be at the first occurence in the text, for example
      ``Large Hadron Collider (LHC)'' and then always used in the acronym form
\end{itemize}	


\section{Completeness of Information}

The information in the thesis shall be complete,
giving the reader all the technical details necessary for the understanding of each reported result.
References to the thesis itself, like "as written before" or such, shall be avoided,
since it's understood that the reader already read what preceeds each sentence,
or is able to browse the table of contents to identify the needed piece of information.
\newline{}
At the same time,
this does not imply that the thesis is a narration of the thesis work:
the focus is on the results, not on the history of their achievement.
For example,
failed attempts shall not be reported if they do not constitute a relevant scientific piece of information.


\section{Images and Tables}

The writer may use tables and figures in the writing,
remembering that each of them should be always mentioned explicitly in the main text with its numbering,
and that each of them should have a caption long and clear enough 
to allow the reader to understand what is presented 
without the need of searching for the reference in the main text.
