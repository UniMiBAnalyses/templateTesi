\chapter{Title of the First Chapter}

A thesis may be ideally thought of as divided into three parts.
The first chapter contains the description of the environment where the work takes place,
both theoretical and experimental.
Both of them are instrumental to the work presented in the thesis, 
and are not meant to replace any notions that the reader would rather acquire 
from the existing literature.
The target audience of this chapter are physicists,
who are well aware of the main principle of the modern physics,
threfore the is meant only to address the specific knowledge to understand the work presented
in the following chapters,
in terms of experimental situation and theoretical motivations.
This may for example include a brief description of the Large Hadron Collider
and of the CMS detector, for the benefit of colleagues working on other subjects
for what concerns the experimental part; 
of the generic particle content of the Standard Model,
its limitations if needed,
and the connection to the aims of the work.

The writer may use tables and figures in the writing,
remembering that each of them should be always mentioned explicitly in the main text with its numbering,
and that each of them should have a caption long and clear enough 
to allow the reader to understand what is presented 
without the need of searching for the reference in the main text.
